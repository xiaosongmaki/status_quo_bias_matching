\documentclass{article}
\usepackage{amsmath}
\usepackage{graphicx}
\usepackage{booktabs}

\title{Analysis of Strategic Behavior in School Choice Mechanisms}
\author{HUANG XING}
\date{\today}

\begin{document}
\maketitle

\section{Results for 3 Students and 3 Schools Scenario}

In this section, I present my analysis of strategic behavior in a 3x3 school choice mechanism implemented over two rounds. My simulation framework systematically explores the possibility of beneficial preference manipulation.

\subsection{Simulation Framework}
My code implements a comprehensive search for strategic manipulation opportunities with the following structure:
\begin{itemize}
    \item Student $s_1$'s true preference is fixed as: $c_1 \succ c_2 \succ c_3 $
    \item For all other agents (3 students and 3 schools), each has 6 possible preference orderings
    \item This creates a total of $6^5=7776$ possible preference combinations to examine
\end{itemize}

\subsection{Strategic Analysis Methodology}
For each preference profile, I investigate whether $s_1$ can benefit from strategic manipulation through:
\begin{enumerate}
    \item First round: $s_1$ misreports preferences while others report truthfully
    \item Second round: 
        \begin{itemize}
            \item Other students update their preferences based on first-round outcomes
            \item $s_1$ reverts to truthful reporting
        \end{itemize}
    \item Comparison: I compare $s_1$'s final outcome under strategic behavior versus truthful reporting
\end{enumerate}

\subsection{Key Findings}
My exhaustive simulation revealed:
\begin{itemize}
    \item Among the $6^5$ total cases examined, 144 cases demonstrated successful strategic manipulation
    \item In these cases, $s_1$'s strategic misreporting in the first round, followed by truthful reporting in the second round, led to a strictly better outcome compared to being truthful throughout
    \item This suggests that even in a relatively simple two-round matching mechanism, there exist significant opportunities for beneficial strategic behavior
\end{itemize}

\subsection{Detailed Case Study of Beneficial Strategic Behavior}
Let's examine a specific case (Case ID 1044) that demonstrates how strategic manipulation can lead to better outcomes:

\subsubsection{Initial Preference Setup}
\begin{itemize}
    \item True Student Preferences:
    \begin{itemize}
        \item $s_1$: $c_1 \succ c_2 \succ c_3$
        \item $s_2$: $c_1 \succ c_2 \succ c_3$
        \item $s_3$: $c_3 \succ c_1 \succ c_2$
    \end{itemize}
    \item School Preferences:
    \begin{itemize}
        \item $c_1$: $s_3 \succ s_2 \succ s_1$
        \item $c_2$: $s_1 \succ s_2 \succ s_3$
        \item $c_3$: $s_1 \succ s_2 \succ s_3$
    \end{itemize}
\end{itemize}

\subsubsection{Truthful Reporting Scenario}
When all students report preferences honestly:
\begin{itemize}
    \item First round matching: $\mu_1 = \{(s_1,c_2), (s_2,c_1), (s_3,c_3)\}$
    \item $s_1$ receives their second choice $c_2$
\end{itemize}

\subsubsection{Strategic Manipulation}
$s_1$ employs the following strategy:
\begin{enumerate}
    \item Misreports first-round preference as: $c_1 \succ c_3 \succ c_2$
    \item First round matching results: $\mu'_1 = \{(s_1,c_3), (s_2,c_2), (s_3,c_1)\}$
    \item Second round preference updates:
    \begin{itemize}
        \item $s_2$ updates to: $c_2 \succ c_1 \succ c_3$
        \item $s_3$ maintains: $c_3 \succ c_1 \succ c_2$
    \end{itemize}
    \item $s_1$ reverts to truthful reporting in second round
    \item Final matching: $\mu'_2 = \{(s_1,c_1), (s_2,c_2), (s_3,c_3)\}$
\end{enumerate}
Through this strategic manipulation, $s_1$ ultimately obtains their top choice $c_1$, a clear improvement over $c_2$ received under truthful reporting. 






\section{Results for 4 Students and 4 Schools Scenario}
% 这里可以添加:
% - 更复杂情况下的偏好描述
% - 扩展情况下的均衡分析
% - 4x4情况的关键发现
% - 相关数据可视化

\section{Intuition Behind First-Round Manipulation}
% 这里可以添加:
% - 理论分析
% - 具体例子说明
% - 为什么第一轮说谎可能带来更好结果的直观解释
% - 可能的博弈论解释

\end{document} 